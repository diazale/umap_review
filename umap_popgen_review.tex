% Use only LaTeX2e, calling the article.cls class and 12-point type.

\documentclass[12pt]{article}

\usepackage{scicite}
\usepackage{lineno}
\linenumbers

% Use times if you have the font installed; otherwise, comment out the
% following line.
\usepackage{times}

% useful packages
\usepackage{xcolor}
%\usepackage{hyperref}
\usepackage[hyphens]{url}
%\PassOptionsToPackage{hyphens}{url}\usepackage{hyperref}
\usepackage{comment}
\usepackage{graphicx}
\usepackage{subcaption}


\topmargin 0.0cm
\oddsidemargin 0.2cm
\textwidth 16cm 
\textheight 21cm
\footskip 1.0cm


\newenvironment{sciabstract}{%
\begin{quote} \bf}
{\end{quote}}


% If your reference list includes text notes as well as references,
% include the following line; otherwise, comment it out.

\renewcommand\refname{References and Notes}

\newcommand{\adpcomment}[1]{{\textcolor{orange}{ADP: #1}}}
\newcommand{\sgcomment}[1]{\textcolor{red}{SG: #1}}

% The following lines set up an environment for the last note in the
% reference list, which commonly includes acknowledgments of funding,
% help, etc.  It's intended for users of BibTeX or the {thebibliography}
% environment.  Users who are hand-coding their references at the end
% using a list environment such as {enumerate} can simply add another
% item at the end, and it will be numbered automatically.

\newcounter{lastnote}
\newenvironment{scilastnote}{%
\setcounter{lastnote}{\value{enumiv}}%
\addtocounter{lastnote}{+1}%
\begin{list}%
{\arabic{lastnote}.}
{\setlength{\leftmargin}{.22in}}
{\setlength{\labelsep}{.5em}}}
{\end{list}}


% Include your paper's title here
\title{A review of UMAP in population genetics} 


% Place the author information here.  Please hand-code the contact
% information and notecalls; do *not* use \footnote commands.  Let the
% author contact information appear immediately below the author names
% as shown.  We would also prefer that you don't change the type-size
% settings shown here.

\author
{Alex Diaz-Papkovich$^{1,2}$, Luke Anderson-Trocm\'{e}$^{2}$, Simon Gravel$^{2\ast}$\\
\\
\normalsize{$^{1}$Quantitative Life Sciences Program, McGill University, Montreal,}\\
\normalsize{$^{2}$Department of Human Genetics, McGill University, Montreal}\\
\\
\normalsize{$^\ast$To whom correspondence should be addressed; E-mail:  simon.gravel@mcgill.ca.}
}

% Include the date command, but leave its argument blank.

\date{}



%%%%%%%%%%%%%%%%% END OF PREAMBLE %%%%%%%%%%%%%%%%



\begin{document} 

% Double-space the manuscript.

\baselineskip24pt

% Make the title.

\maketitle 


% Place your abstract within the special {sciabstract} environment.

\begin{sciabstract}
Uniform manifold approximation and projection (UMAP) has been rapidly adopted by the population genetics community to study population structure. It has become common in visualizing the ancestral composition of human genetic datasets, as well as searching for unique clusters of data, and for identifying geographic patterns. Here we give an overview of applications of UMAP in population genetics, provide recommendations for best practices, and offer insights on optimal uses for the technique.
%and offer visualizations and applications using using publicly available data.
\end{sciabstract}

\section*{Introduction}
% give brief statement on population stratification?

One of the primary challenges of genomic data analysis is high dimensionality. The human genome has over three billion base pairs, and many biobanks contain hundreds of thousands of individuals and above. Relationships among individuals are relevant for historical studies as well as for studies that seek to identify genetic roots of diseases. These relationships can be influenced by demography, by sampling strategies, as well as by technical variation. A first step in many genomic analyses is dimensional reduction approach to visualize the data to identify relevant relatedness patterns. 

One of the most common methods of dimensionality reduction is principal component analysis (PCA) to identify directions, in the high-dimensional space, along which data is most variable. The projection of genomic data along these directions provides a low-dimensional representation of the data that captures as much variance as possible. Because PCA projection is a linear operation, it has a relatively straightforward interpretation in terms of demographic events \cite{mcvean2009genealogical}, and is also well-suited to the correction of population structure in genome-wide association studies (GWAS) \cite{patterson2006population}, it is very widely used.

Dimension reduction requires tradeoffs. Because PCA projection identifies directions of maximal variance in the data and ignores other variance components, it tends to obscure finer scale patterns of variation.  Many nonlinear neighbour graph-based dimension reduction algorithms have been developed over the years to address this need. Here we focus on uniform manifold approximation and projection (UMAP)\cite{mcinnes_umap_2018}, a method developed in 2018 that has seen widespread adoption in many fields (e.g. single-cell genomics\cite{becht_dimensionality_2019}). 

Rather than trying to preserve large-scale structure, UMAP seeks to preserve local neighbourhoods in a dataset. For each individual in a genetic dataset, UMAP identifies a pre-set number of nearest neighbours and represents distances to these neighbours as a weighted graph where nearest neighbours are weighted more heavily. The goal is then to find a low-dimensional representation of the data that preserves these neighbourhoods as much as possible. By focusing on preserving neighborhood topology rather than absolute distances, UMAP allows for data-dense regions to be ``stretched out'' in the representation. This has the benefit of reducing overcrowding of the low-dimensional representation, but comes at the cost of a more challenging interpretation of distances in the low-dimensional representation.  This is an important distinction relative to algorithms such as PHATE \cite{moon2019visualizing} that allow nonlinear transformations of the data while seeking to preserve meaningful distances. 

By focusing on preserving topology rather than absolute distances, UMAP representations can also contain arbitrarily small distances between points. Though such small distances might be a faithful representation of the original data topology, they are not ideal for visualization. UMAP allows for specification of a minimum distance between nearest neighbours in low-dimensional space. This is useful for visualization, but can be set to zero or very small values when using the reduced dimension representation for downstream analyses, such as clustering.

In the context of genetic data, UMAP finds the nearest genetic neighbours for each individual and creates low-dimensional representations that group more closely-related individuals together, and partially preserves longer-range relatedness through intermediary individuals. When used in visualizations, UMAP embeddings uncover many subtle features of data, such as distinct demographic histories and covariation between genetics, geography, and phenotypes\cite{diaz-papkovich_umap_2019}. Figure~\ref{fig:PCA_and_UMAP} compares visualizations of PCA to UMAP using genotype data from the Thousand Genomes Project (1000GP)\cite{10002015global}. PCA flattens the third dimension, obsuring the distinction between South Asian and Central/South American populations, whereas UMAP places them in more clearly visible clusters. UMAP has become widely used to study population structure in humans and other species, in conjunction with existing methods. Here we will describe the current state of the use of UMAP in population genetics.

\section*{Visualizing genomic cohorts}

The most straightforward and common use of UMAP is for visualization. This has proven useful for data composed of relatively homogeneous populations as well as those with considerable diversity in ancestries. UMAP will dedicate more visual space to larger populations within a cohort, and consequently can illustrate the ancestral composition of a cohort in the context of its population structure as well as the size of the data. Often these data are combined with reference panels such as the 1000GP or the Human Genome Diversity Project (HGDP)\cite{cann2002human}.  As with PCA, researchers can either perform the dimensionality reduction jointly or project one dataset onto UMAP embeddings of reference data. In most surveyed literature, data are restricted to common variants with a minor allele frequency (MAF) greater than some threshold, e.g. $0.01$. However, UMAP can be used to identify population structure on any subset of interest.

In the first use of UMAP in population genetics, we demonstrated how it illustrates fine-scale structure and clustering in datasets, and that it is well-suited to data with a high number of significant PCs\cite{diaz-papkovich_umap_2019}.  Figure~\ref{fig:PCA_and_UMAP} uses 1000GP data to show how UMAP emphasizes local structure over PCA's focus on global variation. Because of its sampling of distinct population with comparable sample sizes, the 1000GP visualizations reveal distict groups of equal size. Figures~\ref{fig:UKB_PCA} and \ref{fig:UKB_UMAP} visualize the UK biobank (UKB)\cite{sudlow2015uk} data using PCA and UMAP, respectively, showing how UMAP simultaneously reveals continental and subcontintental ancestries in the context of each population's size and the relationships between them. Figures~\ref{fig:gnomAD_UMAP} and \ref{fig:BBJ_UMAP} visualize, respectively, the Genome Aggregation Database (gnomAD v3) from the Broad Institute\cite{karczewski_mutational_2020} and Biobank Japan (BBJ)\cite{nagai2017overview}\cite{sakaue_dimensionality_2020}, each of which contains over $100,000$ individuals. The former is composed of an ancestrally diverse American population, while the latter is relatively more homogeneous.

%; and the genotypes of the descendents of Transatlantic Slave Trade \cite{micheletti_genetic_2020}
In data with more varied ancestries, such as the UKB and gnomAD, but also Bio\textit{Me}\cite{belbin_towards_2019} and the Million Veterans Program (MVP) \cite{hunter-zinck_genotyping_2020}, UMAP tends to highlight broad ancestral groups and admixture history, whereas in relatively more homogenous populations such as BBJ, it highlights clusters related to geographic features like island populations. 

UMAP has also been successfully used with ancient DNA samples combined with modern and contemporary populations to identify shared population structure\cite{margaryan_population_2019}, as well as animal populations to study spatial introgression in mussels\cite{simon_local_2019}, genetic bottlenecks in the white rhino population\cite{sanchez-barreiro_historical_2020}, and the geographic origin of disease-carrying mosquitoes\cite{consortium_genome_2020}\cite{schmidt_population_2020}. 

In all these applications, data points were colored using categorical variables such as geographic origin or self-reported ancestry to help with interpretation. We have also found it informative to colour visualizations by continuous variables such as geographical coordinates, phenotype values, or global admixture proportions as in \cite{diaz-papkovich_umap_2019}, \cite{dai_population_2020}, and \cite{spear2020recent}.

\section*{Supporting analyses: What do I do with a UMAP projection?}
Within Tukey's paradigm of exploratory data analysis, visualization with UMAP can be one of the first steps to the interrogation of complex data\cite{holmes2018modern}. UMAP is useful for identifying clusters in genetic data when the number of clusters is not known in advance\cite{tonkin-hill_fast_2019}, and when there are a high number of significant PCs\cite{diaz-papkovich_umap_2019}. One straightforward approach is to run UMAP again on a cluster itself to examine subcontinental population structure, as in the National Geographic Genographic Project\cite{dai_population_2020}. One may run UMAP on several types of genetic data; this was the case with Almarri et al's study of structural variants, where they found population stratification in all classes of genetic variants, with Oceanian populations consistently forming their own clusters\cite{almarri_population_2020}. In Spear et al., we identified several clusters of Hispanic/Latinx populations using UMAP on the top PCs, despite these groups having overlapping proportions of continental ancestry proportions, and further studied the Mexican-American population to identify temporal and demographic patterns in their admixture histories\cite{spear2020recent}. In each case, these projections were combined with traditional statistical approaches such as $F_{ST},$ Admixture \cite{}, or fineSTRUCTURE\cite{lawson2012inference}.

One promising application is the use of clusters as covariates in GWAS and polygenic scores (PGS). Fine-scale population structure continues to confound studies of polygenic traits whether in studies of ancestrally diverse or relatively homogeneous populations (e.g. \cite{kerminen2019geographic}, \cite{berg2019reduced}, \cite{sohail2019polygenic}), making it an important area of study. Sakaue et al. used UMAP to identify substructure within the Japanese population, separating it into a mainland population and Hokkaido-Ainu with surrounding islands, reflecting known demographic history in Japan\cite{sakaue_dimensionality_2020}. They identified systematic shifts in PGS for multiple traits across UMAP clusters . 

Data cleaning, including LD thinning, is particularly important when performing UMAP. Certain regions, such as the human leukocyte antigen (HLA) region in the genome, can unduly influence clustering and visualization results. Whereas the influence of HLA might be only observed in a higher-order PC, UMAP can identify the clustering of haplotypes at a single, densely typed locus and represent carriers of that haplotype as a distinct cluster (figure~\ref{fig:HLA}). LD thinning addresses this issue. Thus careful data preparation is particularly important for UMAP, and researchers should resist the tendency to assign a demographic explanation to each cluster. 

Th capacity of UMAP to identify haplotype structure was used by Yamamoto et al use UMAP to visualize mitochondrial DNA (mtDNA). Whereas UMAP correctly identified sub-haplogroup clusters of mitchondrial DNA it did not identify parent clusters as readily as PCA or phylogenetic analysis, and is not particularly advantageous for single-locus analysis. \cite{yamamoto_genetic_2020}. 

\section*{Discussion}
UMAP is now used regularly to visualize the ancestral composition of cohorts as well as to examine fine-scale population structure and subtle patterns in biobanks of all compositions. In this sense, UMAP --- and dimensionality reduction at large --- is to data what a microscope is to biological samples: an effective tool to scientifically examine a subject and provoke deeper investigation. In both cases, calibration is an important factor, as is understanding the tool's limitations. The main parameters to calibrate in UMAP are the number of nearest neighbours (NN) and the minimum distance (MD). Studies varied in their parameter selection, but generally chose NN close to $15$; setting $NN < 10$ can result in separated clusters made up of closely-related individuals, such as families. The minimum distance was usually $0.1 < MD < 0.5$; values of $MD$ close to $0$ create very tight clusters, which can be appropriate for  downstream process such as cluster analysis but less pleasing visually. We recommend running multiple parametrizations and to combine UMAP plots with PCA plots and methods like fineSTRUCTURE\cite{lawson2012inference}, ADMIXTURE\cite{alexander2009fast}, or traditional statistics such as $F_{ST}$ to make inferences. As with PCA and other dimensionality reduction methods, genetically-defined clusters represent some degree of shared ancestry. While genetic clusters correlate with variables like self-identified ethnicity or race, they are distinct concepts and not interchangeable\cite{mathieson2020ancestry}.

The reference implementation of UMAP is regularly updated with new features\cite{mcinnes2018software}. A recent update enabled visualization of the simplicial complex underlying the algorithm, which can highlight how input data and parameterization impact the formation and placement of clusters relative to one another. We demonstrate this using genotype data from the 1000GP in figure~\ref{fig:UMAP_connectivity}. Increasing the value of $NN$ increases the size of the complex (at a higher computational cost), but clusters that are completely disjoint from the rest of the data when $NN=15$ become connected as $NN$ is increased to $200$. In figures~\ref{fig:UMAP_low_NN_1KGP} and \ref{fig:UMAP_low_NN_connectivity} the simplicial complexes of South Asian and East Asian populations do not connect to other populations; that is, for these continental clusters, every individual's $15$ closest genetic neighbours fall within the cluster. In figures~\ref{fig:UMAP_high_NN_1KGP} and \ref{fig:UMAP_high_NN_connectivity}, where $NN=200$, all continental populations become connected. Some populations, such as the Luhya (LWK) and Japanese (JPT), become more closely connected to their continental groups, and the embedding with $NN=200$ places their subclusters closer to their respective continental populations. These visualizations also clarify that, since UMAP preserves these topological connections, the relative positions of connected clusters may be flipped or rotated relative to each other when carrying out multiple runs with identical parameters.

\section*{Conclusion}
With its effective performance and widespread use in under two years, UMAP shows considerable promise as part of the toolbox of a population geneticist, especially in the case of large cohorts. Beyond its capacity to visualize data, it holds promise for downstream methods such as clustering, correction for fine-scale population structure in GWAS and PGS, and identifying unique demographic histories. We anticipate that UMAP and/or related methods of dimensionality reduction will continue to find applications in the field, bolstering our exploration and understanding of human genomic data and the study of complex polygenic traits.

\section*{Materials and methods}
All code used to process 1000GP data and generate images is available at \url{https://github.com/diazale/umap_review}. We used genotype data from $3,450$ individuals from the 1000GP using Affy 6.0 genotyping\cite{10002015global}. Genotype data from the 1000GP is available at \url{http://ftp.1000genomes.ebi.ac.uk/vol1/ftp/release/20130502/supporting/hd_genotype_chip/} and \url{http://ftp.1000genomes.ebi.ac.uk/vol1/ftp/phase3/}. Visualizations were done with matplotlib\cite{Hunter2007} and PCA was done using sklearn\cite{scikit-learn}.

%%%%% FIGURES %%%%%

\clearpage

\begin{figure}[h!]
  \centering
  \begin{subfigure}[b]{0.45\linewidth}
    \includegraphics[width=\linewidth]{code/images/1KGP_PCA.png}
    \caption{}
    \label{fig:PCA}
  \end{subfigure}
  \begin{subfigure}[b]{0.45\linewidth}
    \includegraphics[width=\linewidth]{code/images/1KGP_genotype_UMAP.png}
    \caption{}
    \label{fig:UMAP}
  \end{subfigure}
  \caption{Visualizations of data from the 1000GP. The first two principal components (left) versus a two-dimensional UMAP embedding (right).     ACB, African Caribbean in Barbados;
    ASW, African Ancestry in Southwest US;
    BEB, Bengali;
    CDX, Chinese Dai;
    CEU, Utah residents with Northern/Western European ancestry;
    CHB, Han Chinese;
    CHS, Southern Han Chinese;
    CLM, Colombian in Medellin, Colombia;
    ESN, Esan in Nigeria;
    FIN, Finnish;
    GBR, British in England and Scotland;
    GWD, Gambian;
    GTH, Gujarati;
    IBS, Iberian in Spain;
    ITU, Indian Telugu in the UK;
    JPT, Japanese;
    KHV, Kinh in Vietnam;
    LWK, Luhya in Kenya;
    MSL, Mende in Sierra Leone;
    MXL, Mexican in Los Angeles, California;
    PEL, Peruvian;
    PJL, Punjabi in Lahore, Pakistan;
    PUR, Puerto Rican;
    STU, Sri Lankan Tamil in the UK;
    TSI, Tuscani in Italy;
    YRI, Yoruba in Nigeria }
  \label{fig:PCA_and_UMAP}
\end{figure}

\clearpage

\begin{figure}[h!]
  \centering
  \begin{subfigure}[b]{0.49\linewidth}
    \includegraphics[width=\linewidth]{code/ukb/images/ukbb_pca_coords_eth.png}
    \caption{A PCA projection of the UKB data.}
    \label{fig:UKB_PCA}
  \end{subfigure}
  \begin{subfigure}[b]{0.49\linewidth}
    \includegraphics[width=\linewidth]{code/ukb/images/UKBB_UMAP_PC20_NC2_NN15_MD05_2018454111_eth.png}
    \caption{UMAP on the top 20 PCs of the UKB.}
    \label{fig:UKB_UMAP}
  \end{subfigure}
  \caption{PCA and UMAP projections of the UKB data, coloured by self-identified ethnic background. Unlike PCA, UMAP focuses on preserving local relationships and emphasizes fine-scale patterns in data. Groups in the UMAP projection are less compressed showing, for example, the relative size of the British and Irish populations in the UKB, alongside populations of other ancestries, while simultaneously showing the population structure between and within groups.}
  \label{fig:UKB}
\end{figure}

\clearpage

\begin{figure}[h!]
  \centering
  \begin{subfigure}[b]{0.49\linewidth}
    \includegraphics[width=\linewidth]{external_images/gnomAD_umap.png}
    \caption{gnomADv3 data visualized using UMAP.}
    \label{fig:gnomAD_UMAP}
  \end{subfigure}
  \begin{subfigure}[b]{0.49\linewidth}
    \includegraphics[width=\linewidth]{external_images/BBJ_UMAP.png}
    \caption{BBJ data visualized using UMAP.}
    \label{fig:BBJ_UMAP}
  \end{subfigure}
  \caption{The Genome Aggregation Database (gnomAD, left) and Biobank Japan (BBJ, right) visualized using UMAP. UMAP illustrates the ancestral diversity of gnomAD, showing many the relationships between populations on continental and subcontinental levels. For the relatively more homogeneous BBJ data, it splits data geographically into the large mainland cluster (consisting of Hokkaido, Tohoku, Kanto-Koshinetsu, Chubu-Hokuriku, Kinki, and Kyushu regions), and smaller non-mainland clusters.}
  \label{fig:external_UMAP}
\end{figure}

\clearpage

\begin{figure}[h!]
  \centering
    \includegraphics[width=\linewidth]{external_images/umap_hla_comparison_highlighted.jpg}
  \caption{UMAP with (left) and without (right) HLA regions used on the Genizon database. The cluster in the dotted lines disappears when filtering for HLA and linkage disequilibrium.}
  \label{fig:HLA}
\end{figure}

\clearpage

\begin{figure}[h!]
  \centering
  \begin{subfigure}[b]{0.48\linewidth}
    \includegraphics[width=\linewidth]{code/images/1KGP_genotype_UMAP_low_NN.png}
    \caption{UMAP with 15 neighbours.}
    \label{fig:UMAP_low_NN_1KGP}
  \end{subfigure}
  \begin{subfigure}[b]{0.48\linewidth}
    \includegraphics[width=\linewidth]{code/images/UMAP_connectivity_low_NN.png}
    \caption{Connectivity map of 15 neighbours.}
    \label{fig:UMAP_low_NN_connectivity}
  \end{subfigure}
  \begin{subfigure}[b]{0.48\linewidth}
    \includegraphics[width=\linewidth]{code/images/1KGP_genotype_UMAP_high_NN.png}
    \caption{UMAP with 200 neighbours.}
    \label{fig:UMAP_high_NN_1KGP}
  \end{subfigure}
  \begin{subfigure}[b]{0.48\linewidth}
    \includegraphics[width=\linewidth]{code/images/UMAP_connectivity_high_NN.png}
    \caption{Connectivity map of 200 neighbours.}
    \label{fig:UMAP_high_NN_connectivity}
  \end{subfigure}
  \caption{UMAP projection of the same genotype data from the 1000GP comparing parametrization with a small (top) and large (bottom) number of nearest neighbours. Left images are coloured by population; right images are the same points but with the simplicial complex drawn. When adding more neighbours, subclusters become less separated, as with the LWK population, for example. Looking at the connectivity maps, we see new connections between continental groups, such as the Central/South American clusters and East Asian clusters. Darker lines indicate that individuals are closer to each other in genotype space.}
  \label{fig:UMAP_connectivity}
\end{figure}

\clearpage
\newpage

\bibliography{umap_popgen_review}

\bibliographystyle{Science}




\end{document}